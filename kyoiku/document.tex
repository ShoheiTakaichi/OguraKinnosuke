\documentclass[11pt,a4paper]{jsarticle}
%
\usepackage{amsmath,amssymb}
\usepackage{bm}
\usepackage{graphicx}
\usepackage{ascmac}
%
\setlength{\textwidth}{\fullwidth}
\setlength{\textheight}{39\baselineskip}
\addtolength{\textheight}{\topskip}
\setlength{\voffset}{-0.5in}
\setlength{\headsep}{0.3in}
%
\newcommand{\divergence}{\mathrm{div}\,}  %ダイバージェンス
\newcommand{\grad}{\mathrm{grad}\,}  %グラディエント
\newcommand{\rot}{\mathrm{rot}\,}  %ローテーション
%
\pagestyle{myheadings}

%\markright{\footnotesize \sf 物理のかぎしっぽ \ \texttt{ http://www12.plala.or.jp/ksp/}}

\title{数学教育の根本}

\begin{document}
\maketitle
%
%
\section{日本の数学教育の現状}

\subsection{日本の数学教育}
\subsubsection{数学教育に関する疑問}
A「数学とはなんだ。」

B「それは中学時代(現在の高等学校)に学校で一番いじめられ、また入学試験でも一番苦しめられた学科のことさ。」

A「そんなに苦しんで習った学科なら、今でも忘れずにいるだろう。」

B「なに、みな忘れてしまった。今でも知ってるのは、ただ方程式とかシムソンの定理とかいう言葉だけさ。」

A「そんなに忘れてしまっても、君は普段何かの拍子に残念なことをしたと思うことはないかね。」

B「ちっともないね。数学なんて、小学校やった加減乗除さえ知っていれば十分だよ。今考えてみると何のためにあんなもので中学生を苦しめるのか、僕はちっとも解らん。何にしろあんな能率の上がらぬものは、またと世界にはあるまいよ。」

これが数学に対する一般人の代表的感想あろうと思う。

なぜに生徒は数学で苦しめられるのか、なぜに学校を出れば忘れてしまうのか、なぜに数学はほとんど生活と無交渉なのか、なぜに学校では能率の上がらない数学をおしえるのか。

今一歩深く考えてみる。数学教育は今日学校で重んじているほど、果たしてそれほど重大な価値を有するのだろうか。数学が生活とほとんど無交渉なのも、能率が上がらぬものとの定評は、はたして正しい判断なのであろうか。それとも今日の数学教育が邪道に陥っていて、教えるべきものを教えず、教えるべからざるものを教えているのだろうか。---ここに現代数学教育に関する根本的疑問がある。

この疑問の解決は決して簡単なことではない。これが解決されるときが、すなわち数学教育の根本問題が解決されるときである。私はこれより順を追って、読者とともにこの問題の解決を試みたいと思う。

\subsubsection{数学教育の特徴}
従来我が国の小学校では算術を、中学校では算術、代数、平面幾何(5年では代数、立体幾何、三角法)を、高等学校では代数、三角法、立体幾何、解析幾何、微積分を教えてきた。我々はこれよりしばらくその教材がいかなるものであるかを述べてみたいと思う。



%
%
\end{document}
