\documentclass[11pt,a4paper]{jsarticle}
%
\usepackage{amsmath,amssymb}
\usepackage{bm}
\usepackage{graphicx}
\usepackage{ascmac}
%
\setlength{\textwidth}{\fullwidth}
\setlength{\textheight}{39\baselineskip}
\addtolength{\textheight}{\topskip}
\setlength{\voffset}{-0.5in}
\setlength{\headsep}{0.3in}
%
\newcommand{\divergence}{\mathrm{div}\,}  %ダイバージェンス
\newcommand{\grad}{\mathrm{grad}\,}  %グラディエント
\newcommand{\rot}{\mathrm{rot}\,}  %ローテーション
%
\pagestyle{myheadings}

%\markright{\footnotesize \sf 物理のかぎしっぽ \ \texttt{ http://www12.plala.or.jp/ksp/}}
\title{図計算および図表}

\begin{document}
\maketitle

私は大正11年(1992)東京物理学校夏季巡回講習会の際, 東京, 高松および鳥取の三市において各々八時間にわたり「図計算と図表」について講義する機会を得ました. その抗議筆記を基といたしまして, 趣旨を変えない程度に老いて改訂増補を加えたのがすなわちこの書であります.

本書は専門家のために書かれたものではございません. 私は一方においては数学の学生と中等学校数学教授者諸兄のために, 他方においては物理学, 化学, 測量より工学, 医学, 理財学, 保険学, 陸海軍等に至るまで, すなわち自然科学より応用科学に至るまでの方々の参考のために, 斯学の輪郭を説明しようと試みたのです.

この目的のために, 私は材料を最小限度に留めまして, その範囲内において, でき得る丈だけ平易に通俗にしかしも懇切に説明しようと企てました. それで高尚な問題と複雑な事柄と厳格な理論とは, 全くこれをすて, しまいました. 解放が多くある場合には, そのうち最も代表的なものただ一つを選びました. また精密な結果を得る方法でも, 作図の複雑なものはこれを捨て, 結果が多少不精密でも作図の簡単なものを採用いたしました. この点において本書は不完全なものでありますけれども, 同時に読みやすくわかりやすいことだけは保証し得ると思われます.


%
%
\section {一変数の整多項式}
\subsection {ゼーグナーの作図法}

一変数の整多項式を
\begin{equation}
y = a_0 x^n + a_1 x^{n-1} + a_2 x^{n-2} + \cdots + a_{n-1} x + a_n
\end{equation}

と致します. ここに$a_0, a_1, \cdots,a_{n-1},a_n$は実定数で, $x$は実変数です. $x$に一つの値を与えるとき, $y$の値を作図のみによって求めること, これが本設の目的であります. 

まずゼーグナー (Segner) が1761年にロシアのペトログラードの学会で発表した方法を説明いたしましょう. 

今簡単のために, 上の式において$n=4$の場合. すなわち
\begin{equation}
y = a_0x^4+a_1x^3+a_2x^2+a_3x+a_4
\end{equation}
をとることにいたします.
けれどもこの方法は$n$の如何にかかわらず, 同様に実行し得られるのであります。

縦線$CO$を引き, その上に点$C_0, C_1, C_2, C_3$を取り (第1図), (長さについては第三疑の注意2をご覧なさい), 
\begin{equation}
CC_0 = a_0, C_0C_1=a_1, C_1C_2=a_2, C_2C_3=a_3, C_3=O=a_4
\end{equation}
なるようにします. また、$O$を過ぎりては$CO$に垂線を引き, その上に長さの単位を$1$に等しく$OU$を取り, また$x$に等しく$OX$を取ります.

このに定数$a_0, a_1, a_2, a_3, a_4$は正ならば下向きに, 負ならば下向きに, また$x$は正ならば$O$より右の方に (すなわち単位を取りたる方), 負ならば左の方に計るのです.

次に第1図において$U$および$X$より$OU$に垂線を立て, $C$より$OU$に平行に引いた直線と$U$における垂線との交点を$Q_0$とし, 直線$C_0Q_0$と$X$における垂線との交点を$P_1$とします. 
次に$P_1$より$OU$に平行に引いた直線と$UQ_0$との交点を$Q_1$とし, $C_1Q_1$と$X$における垂線の交点を$P_2$とします. 同様の作図を繰り返して, 終わりに第1図の如く$P_4$を得ます. このようにいたしますと, $XP_4$は$x$が$OX$のときの$y$の値であります.

\subsection{その証明}
さて、これを証明するにあって、まず次の補題から始めましょう.

直角三角形$ABC$において, $AB=1, BC=a$とします(第3図). $AB$上に一点$D$を取り、$AD=x$とし, $D$より$BC$に平行線DEを引きますと,
\begin{equation}
\frac{AB}{BC}=\frac{AD}{DE}, ゆえに DE = \frac{(BC)(AD)}{AB}=\frac{ax}{1}
\end{equation}
すなわち、$DE$の長さは$ax$に等しいことがわかりました.

この簡単な補題に基づいて, 我々はゼーグナーの方法が正しいことを証明することができるのです.

再び第1図に立ち帰りまして, $C_0, C_1, C_2, C_3$より$OU$に平行線を引き, これらの直線$UQ_0, XP_1$と交わる点をそれぞれ$R_1, R_2, R_3, R_4$および$S_1,S_2,S_3,S_4$, と致します (第4図).

さて直角三角形$C_0R_1Q_0$において

\begin{eqnarray}
P_1S_1&=&Q_0R_1x(補題によりて\\
&=&CC_0x=a_0x
\end{eqnarray}

次に直角三角形$C_1R_2Q_1$において

\begin{eqnarray}
P_2S_2&=&Q_1R_2x(補題によりて\\
&=&(Q_1R_1+R_1R_2)x\\
&=&(P_1S_1+C_0C_1)x=(a_0+a_1)x\\
&=&a_0x^2+a_1x
\end{eqnarray}

更に直角三角形$C_2R_3Q-2$において


%
%
\end{document}
