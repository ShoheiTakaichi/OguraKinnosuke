\documentclass[11pt,a4paper]{jsarticle}
%
\usepackage{amsmath,amssymb}
\usepackage{bm}
\usepackage{graphicx}
\usepackage{ascmac}
%
\setlength{\textwidth}{\fullwidth}
\setlength{\textheight}{39\baselineskip}
\addtolength{\textheight}{\topskip}
\setlength{\voffset}{-0.5in}
\setlength{\headsep}{0.3in}
%
\newcommand{\divergence}{\mathrm{div}\,}  %ダイバージェンス
\newcommand{\grad}{\mathrm{grad}\,}  %グラディエント
\newcommand{\rot}{\mathrm{rot}\,}  %ローテーション
%
\pagestyle{myheadings}

%\markright{\footnotesize \sf 物理のかぎしっぽ \ \texttt{ http://www12.plala.or.jp/ksp/}}
\title{図計算および図表}

\begin{document}
\maketitle
%
%
\section {一変数の整多項式}
\subsection {ゼーグナーの作図法}

一変数の整多項式を
\begin{equation}
y = a_0 x^n + a_1 x^{n-1} + a_2 x^{n-2} + \cdots + a_{n-1} x + a_n
\end{equation}

と致します. ここに$a_0, a_1, \cdots,a_{n-1},a_n$は実定数で, $x$は実変数です. $x$に一つの値を与えるとき, $y$の値を作図のみによって求めること, これが本設の目的であります. 

まずゼーグナー (Segner) が1761年にロシアのペトログラードの学会で発表した方法を説明いたしましょう. 

今簡単のために, 上の式において$n=4$の場合. すなわち
\begin{equation}
y = a_0x^4+a_1x^3+a_2x^2+a_3x+a_4
\end{equation}
をとることにいたします.
けれどもこの方法は$n$の如何にかかわらず, 同様に実行し得られるのであります。

%
%
\end{document}
