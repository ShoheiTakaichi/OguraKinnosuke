私は大正11年(1992)東京物理学校夏季巡回講習会の際, 東京, 高松および鳥取の三市において各々八時間にわたり「図計算と図表」について講義する機会を得ました. その抗議筆記を基といたしまして, 趣旨を変えない程度に老いて改訂増補を加えたのがすなわちこの書であります.

本書は専門家のために書かれたものではございません. 私は一方においては数学の学生と中等学校数学教授者諸兄のために, 他方においては物理学, 化学, 測量より工学, 医学, 理財学, 保険学, 陸海軍等に至るまで, すなわち自然科学より応用科学に至るまでの方々の参考のために, 斯学の輪郭を説明しようと試みたのです.

この目的のために, 私は材料を最小限度に留めまして, その範囲内において, でき得る丈だけ平易に通俗にしかしも懇切に説明しようと企てました. それで高尚な問題と複雑な事柄と厳格な理論とは, 全くこれをすて, しまいました. 解放が多くある場合には, そのうち最も代表的なものただ一つを選びました. また精密な結果を得る方法でも, 作図の複雑なものはこれを捨て, 結果が多少不精密でも作図の簡単なものを採用いたしました. この点において本書は不完全なものでありますけれども, 同時に読みやすくわかりやすいことだけは保証し得ると思われます.
